\documentclass{article}
\usepackage[UTF8]{ctex}
\usepackage{fancyhdr}
\usepackage[ruled,vlined]{algorithm2e}
\usepackage{fontspec}
\usepackage{amsmath}
\usepackage[a4paper, includehead]{geometry}
\geometry{
  left=27mm,
  right=27mm,
  top=25.4mm, 
  bottom=25.4mm,
  heightrounded,
}
\usepackage[style=numeric,
            sorting=none,
            bibstyle=gb7714-2015,
            citestyle=gb7714-2015]{biblatex}
\bibliography{ref.bib}

\usepackage{graphicx}
\usepackage{tabularx}
\usepackage{multirow}
\usepackage{colortbl}
\usepackage[colorlinks,linkcolor=black] {hyperref}

\renewcommand{\contentsname}{\centering {目录}}
\renewcommand{\today}{\number\year 年\number\month 月 \number\day 日}
\newcommand{\mytitle}{报告标题}
\newcommand{\mysubtitle}{报告副标题}

\linespread{1.5}
\lhead{\mysubtitle}
\rhead{\mytitle}
\cfoot{\thepage}

\begin{document}
\pagestyle{empty}
\vskip 40mm
\begin{center}
    \includegraphics[width=0.65\paperwidth]{assets/zjuchar.pdf} \\
    \vskip 10mm
    \centering
    \includegraphics[width=0.3\paperwidth]{assets/zju.pdf}
\end{center}
\vskip 10mm
\begin{center}
    \zihao{-1} \textbf{\mytitle} \\
    \zihao{2} \textbf{\mysubtitle}
\end{center}
\vskip 10mm
\begin{center}
    \zihao{3}
    \begin{tabularx}{.7\textwidth}{>{\fangsong}l >{\fangsong}X<{\centering}}
        姓名与学号 & \uline{\hfill 姓名 12345678 \hfill} \\
                %    & \uline{\hfill 如果多个人,可以再加 \hfill} \\
        年级与专业 & \uline{\hfill 2019级 计算机科学与技术 \hfill} \\
    \end{tabularx}\\
    \vskip 40mm
    \today
\end{center}


\clearpage
\tableofcontents
\clearpage

%%%%%%%%%%%%
% 正文
%%%%%%%%%%%%

\pagestyle{fancy}
\setcounter{page}{1}

\section{H1}

这里是H1下的内容

\subsection{H2}

这里是H2下的内容

\subsubsection{H3}

这里是H3下的内容

这里是对图\ref{fig:zju}的引用

这里是一个引用\cite{崔迪2017面向可视化系统设计与开发的嵌套增量模型}

\begin{enumerate}
    \item 列举1
    \item 列举2
    \item 列举3
\end{enumerate}

\begin{figure}
    \centering
    \includegraphics[scale=0.5]{assets/zju_logo.png}
    \caption{图片标题}
    \label{fig:zju}
\end{figure}

%%%%%%%%%%%%%%%%%%%%%%%%%%

\clearpage

% 如果需要显示没有引用到的才看文献,使用\nocite{}
\nocite{梅鸿辉2016一种全球尺度三维大气数据可视化系统}
\printbibliography

\end{document}